% YAAC Another Awesome CV LaTeX Template
%
% This template has been downloaded from:
% https://github.com/darwiin/yaac-another-awesome-cv
%
% Author:
% Christophe Roger
%
% Template license:
% CC BY-SA 4.0 (https://creativecommons.org/licenses/by-sa/4.0/)
%Section: Work Experience at the top
\sectionTitle{Berufserfahrung}{\faSuitcase}
%\renewcommand{\labelitemi}{$\bullet$}
\begin{experiences}
  \experience
  {Derzeit}{Expert Data Scientist | Machine Learning Engineer}{WidasConcepts GmbH}{Wimsheim, Germany}
  {Sep 2018} {
    \vspace{0.1cm}
    \textbf{Project}: Cidaas-ID-Validator (2018 -- Present) \newline
    \textbf{Role}: Expert Data Scientist
    \begin{itemize}
      \item Entwicklung verschiedener Microservices für die Bereitstellung von Machine Learning-Modellen
      \item Entwurf und Entwicklung eines OCR-Modells für den Abruf von Informationen aus Identitätsdokumenten
      \item Entwicklung eines Sicherheitskontrollsystems für Identitätsdokumente unter Verwendung von Computer Vision und probabilistischen grafischen Modellen
      \item Entwurf und Implementierung einer HCI-Anwendung zur Erkennung von Liveness unter Verwendung der Eye-Tracking-Methode (ninoxipy)
      \item Mitwirkung an der Schaffung eines MLOps-Rahmens für die eingeführten Modelle
    \end{itemize}
    \vspace{0.1cm}
    \textbf{Project}: Cidaas-FDS (2020 -- Present)\newline
    \textbf{Role}: Lead Data Scientist
    \begin{itemize}
      \item Entwurf einer ML-basierten Cybersicherheitsarchitektur
      \item Implementierung von RESTful APIs für Modelle zur Betrugserkennung
      \item Mitwirkung an der Erstellung einer CI/CD-Pipeline für die Bereitstellung und Prüfung von Modellen
      \item Entwurf und Entwicklung einer intelligenten MFA mit Fuzzy Inference System (FIS)
    \end{itemize}
    \vspace{0.1cm}

    \textbf{Project}: Bosch-EBR (2020 -- Present)\newline
    \textbf{Role}: Senior Data Scientist and Senior Developer
    \begin{itemize}
      \item Entwicklung eines NLP-basierten Fahrzeugidentifikationssystems unter Verwendung von NER und Graphentheorie
      \item Implementierung einer RESTful API zur Bereitstellung von Fahrzeugidentifikationsmodellen
      \item Entwicklung eines Stream-Processing-Frameworks mit Faust für EBR-Microservices
    \end{itemize}
    \vspace{0.1cm}

    \textbf{Project}: Porsche-CCD (2018 -- 2019)\newline
    \textbf{Role}: Data Scientist
    \begin{itemize}
      \item Entwurf und Entwicklung eines räumlich-zeitlichen Modells der internen sensorischen Daten und der Geolokalisierung von Porsche-Fahrzeugen zur Erkennung von Kurvenfahrten
      \item Erstellen einer interaktiven Benutzeroberfläche zur Überwachung von Porsche-Fahrzeugen und plausiblen Eckfällen
      \item Entwicklung eines Datenstreaming-Systems mit MQTT und Tornado
    \end{itemize}
    \vspace{0.1cm}

    \textbf{Project}: Chinesische Malerei Siegel Bewertung (2018 -- 2019)\newline
    \textbf{Role}: Data Scientist
    \begin{itemize}
      \item Entwurf und Entwicklung eines Deep-Learning-Modells zur Identifizierung von chinesischen Künstlersiegeln
      \item Entwurf und Entwicklung eines wiederkehrenden Modells zur Überprüfung der Originalität der Siegel der Künstler
    \end{itemize}
  }
  {Docker, Kubernetes, GitLab, DC/OS, Python, C++, R,  OpenCV, dlib, Ray, ReactJS, Elasticsearch, Redis, Mongodb, Apache Kafka, spacy, nltk, tesseract, ocrupy, FastAPI, Flask, tornado, Jupyter-notebook, numpy, scipy, pandas, matplotlib, seaborn, bokeh, PyTorch, Pyro, Tensorflow, keras, scikit-learn, scikit-image, scikt-fuzzy}
  \emptySeparator
  \experience
  {Aug 2018}{Researcher and Developer}{DFKI GmbH}{Kaiserslautern, Germany}
  {Sep 2014} {
    \begin{itemize}
      \item Modellierung von Blickbewegungen für Human-Document Interaction für das Immersive Quantified Learning Lab (iQl)
      \item Entwurf eines Mensch-Computer-Interaktionssystems auf der Grundlage der Eye-Tracking-Technologie zur Unterstützung von Behinderten (AICASys-Projekt)
      \item Entwicklung eines neuen generativen probabilistischen Modells zur Synthese von Augenbewegungsmustern
      \item Lehrtätigkeit an der TU Kaiserslautern im Studiengang Data Mining als Lehrbeauftragter
      \item SBetreuung von Master- und Bachelor-Arbeiten im Bereich KI und HCI an der TU Kaiserslautern

    \end{itemize}
  }
  {Docker, GitLab, Python, C++, R, Matlab, JavaScript, ZMQ, IoT, OpenCV, dlib, jupyter-notebook, numpy, scipy, pandas, matplotlib, seaborn, bokeh, Edward, keras, scikit-learn, scikit-image, scikt-fuzzy, Git}
  \emptySeparator
  \experience
  {Aug 2014}{Data Scientist | Web Developer}{Freelancer}{Tehran, Iran}
  {Nov 2013} {
    \begin{itemize}
      \item Entwicklung einer ML-basierten Persönlichkeitsbewertung für die psychologische Bewertung der Mitarbeiter der MAPNA-Gruppe (Iran)
      \item Entwurf und Entwicklung eines Portals für psychologische Gutachten mit PHP und Python
    \end{itemize}
  }
  {Python2, PHP, XAMPP, JavaScript, CSS, Autobahn|Python, TortoiseSVN, plotly, spyder}
  \emptySeparator
  \experience
  {Nov 2013}{University Lecturer}{Azad University}{Tehran (Roudehen), Iran}
  {Sep 2012} {
    \begin{itemize}
      \item Lehrtätigkeit in den Kursen Statistik, Data Mining und Bildverarbeitung für Studenten der Informatik im Grundstudium

      \item Lehrtätigkeit in den Kursen Explorative Datenanalyse und Inferenzstatistik für Postgraduierte der Klinischen Psychologie
    \end{itemize}
  }
  {Python, R, SPSS, Matlab}
\end{experiences}
